\section{Introduction}
\begin{itemize}
	\item ACS: The ALMA Common Software (ACS) provides a software infrastructure common to all partners and consists of a documented collection of common patterns 
		and of software, which implements those patterns. The heart of ACS is based on a distributed component model, with ACS components implemented as CORBA
		objects in any of the supported programmin languages. The teams responsible for the control system development use ACS Components as the basis for 
		high level control entities and for the implementation of devices, such as an antenna mount control. ACS provides common CORBA-based services such as 
		logging as well as depleyment, error and alarm management, configuration database and lifecycle management.
		\url{http://www.eso.org/~almamgr/AlmaAcs/index.html}
	\item  ALMA: The Atacama Large Millimeter Array (ALMA) is a joint project involving astronomical organizations in Europe, North America and Japan. ALMA will 
		consist of 66 antennas operating in the millimeter and sub-millimeter wavelength range, with baselines of more than 10 km. It will be located at an 
		altitude above 5000m in the Chilean Atacama desert. The ALMA Computing group is a joint group with staff scattered on 3 continents and is responsible 
		for all the control and data flow software related to ALMA, including tools ranging from support of proposal preparation to archive access of 
		automatically created images. Early in the project it was decided that an ALMA Common Software (ACS) would be developed as a way to provide a common
		software platform to all 
		partners involved in the development a common software platform. The original assumption was that some key middleware communication via CORBA and 
		the use of XML and JAVA would be part of the project. It was intended from the beginning to develop this software in an incremental way based on 
		releases, so that it would then evolve into an essential embedded part of all ALMA software applications.
		\url{http://www.almaobservatory.org/en}
	\item OpenSource: Although designed for ALMA, ACS can and is being used in other control systems and distributed software projects, since it implements proven 
		design patterns using state of the art, reliable technology. Through the use of standard constructs and components, non-ACS developers can easily 
		understand the architecture of software modules. This makes maintenance affordable even on a very large project such as ALMA. ACS is publicly 
		available under the GNU LGPL license.
		\url{http://www.gnu.org/copyleft/gpl.html}
\end{itemize}

