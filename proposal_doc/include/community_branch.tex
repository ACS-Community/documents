\section{Community Branch}
\begin{itemize}
	\item Licenses: ACS has been developed under the LGPL, but some of the external tools, which aren't required for a basic installation use other licenses such as GPL. For this reason it is of up-most importance to determine which pieces of software, that are not under the LGPL are required, and how to distribute them without reducing the freedom of the users of ACS to distribute their software as they intend.
	\item Technical Requirements:
		\begin{itemize}
			\item Who will be able to propose:
			\begin{itemize}
				\item (1): Current Community: The current community will be the set of projects that are based on ACS and that are 
				actively working with it.
				\item (2): Potential Users: The potential users are the projects that have shown some sort of interest in using ACS.
				\item (3): Interested people: The interested people are individuals that might be willing to contribute with ACS.
			\end{itemize}
			\item How will the proposals be made:
			\begin{itemize}
				\item (1): There should be a platform for tracking requests/proposals and the work being done.
				\item (2)(3): There should be either an open side of the platform for public requests/proposal, or direct contact 
					by e-mail or similar method. 
			\end{itemize}
			\item How will the priorities be assigned: There are a lot of variables that could affect the decision to assign the priorities, making this a non-trivial problem. We need to discuss this with the community before deciding an adequate mechanism.
		\end{itemize}
	\item Participation policies
		\begin{itemize}
			\item Who will work on the requirements
			\begin{itemize}
				\item (1): Main developers: This will be the development manpower from the group in charge of maintaining ACS.
				\item (2): Current community developers: This will be the current community manpower destined to work on 
				ACS from the set of projects that are based on ACS and that are actively working with it.
				\item (3): Volunteers: This will be volunteer work done by interested people.
				\item (4): ACS patches from ALMA: This will be a back-port effort from ALMA's official repository fixes and improvements of ACS.
			\end{itemize}
			\item How will the development be done
			\begin{itemize}
				\item (1): The work is implemented and documented by the developer, along with suitable test cases and directly committed 
					to the respository.
				\item (2): The work is implemented and documented by the developer, along with suitable test cases. Validated by a main 
					developer and directly (or indirectly) committed to the respository.
				\item (3): The work is implemented by the volunteer, who shall send a patch to the main developers. The developers should 
					validate, document and add test cases to the contribution (if the volunteer didn't provide them) and, finally, 
					commit to the repository.
				\item (4): The work is implemented by ALMA developers. The main developers should obtain the patches, resolve the 								(possible) issues and document the contribution.
			\end{itemize}
			\item Tools
			\begin{itemize}
				\item Documentation: TWiki, GitHub wiki. is a flexible, powerful, and easy to use enterprise wiki, enterprise collaboration platform, 
					and web application platform. It is a Structured Wiki, typically used to run a project development space, a document 
					management system, a knowledge base, or any other groupware tool, on an intranet, extranet or the Internet. 
				\item Task Tracking: JIRA (JIRA has a free version for OpenSource projects): JIRA sits at the center of your development team, 
					connecting the people and the work being done. Track bugs and tasks, link issues to related source code, 
					plan agile development, monitor activity, report on project status, and more. 
				\item Lean Manufacturing: Kanban plugin for JIRA. GreenHopper is an add-on for JIRA that facilitates Agile project management. 
					GreenHopper leverages JIRA's flexible workflow to fit your teams Agile process and the workflow can adapt as your team evolves.
				\item Periodical builds: Jenkins. Jenkins provides an easy-to-use so-called continuous integration system, making it easier 
					for developers to integrate changes to the project, and making it easier for users to obtain a fresh build. The automated, 
					continuous build increases the productivity. Also to ensure the status of code about different supported platforms.
				\item Periodical testing: Jenkins. To see the system stability.
			\end{itemize}
		\end{itemize}
\end{itemize}




