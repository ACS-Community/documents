\section{Community}
Current Users: ACS is currently used for the following list of observatories:
\begin{itemize}
	\item The Cherenkov Telescope Array: The CTA project is an initiative to build the next generation ground-based very high energy gamma-ray instrument. It 
		will serve as an open observatory to a wide astrophysics community and will provide a deep insight into the non-thermal high-energy universe.
		\url{http://www.cta-observatory.org/}
	\item Gran Telescopio Canarias: GTC  is a 10.4m telescope with a segmented primary mirror. It is located in one of the top astronomical sites in the Northern 
		Hemisphere: the Observatorio del Roque de los Muchachos (ORM, La Palma, Canary Islands). The GTC is a Spanish initiative leaded by the Instituto de 
		Astrofísica de Canarias (IAC). The project is actively supported by the Spanish Government and the Local Government from the Canary Islands through 
		the European Funds for Regional Development (FEDER) provided by the European Union.
		\url{http://www.gtc.iac.es/en/}
	\item Sardinia Radio Telescope: The Sardinia Radio Telescope is a large, fully steerable radio telescope currently being completed near San Basilio, province 
		of Cagliari in Sardinia, Italy. It is a collaboration among the Istituto di Radioastronomia di Bologna, the Cagliari Astronomy Observatory (Cagliari) 
		and the Arcetri Astrophysical Observatory (Florence). It has been completed in 2011.
		\url{http://www.srt.inaf.it/}
	\item Telescopio Nazionale Galileo: is a 3.58m Italian telescope located on the island of San Miguel de La Palma (or, more simply, La Palma), in the Canary 
		Islands archipelago. It is one of the largest telescopes hosted by the Roque de los Muchachos Observatory, a very important observing site in the 
		northern hemisphere. It is now operated by the "Fundación Galileo Galilei, Fundación Canaria", a no-profit institution which manages the telescope 
		on behalf of INAF, the Italian National Institute of Astrophysics. The telescope saw its first light in 1998. 
		\url{http://www.tng.iac.es/}
	\item Hexapod Telescope: The Hexapod-Telescope (HPT) is a telescope located at Cerro Armazones Observatory in northern Chile. The 1.5 metres (59 in) 
		Ritchey-Chrétien reflecting telescope is notable for the design of telescope mount. Instead of the typical mounting where the telescope moves on two 
		rotating axes, the mirror end of the telescope is supported by six extensible struts, an arrangement known as a Stewart platform. This configuration 
		allows the telescope to move in all six spatial degrees of freedom and also provides strong structural integrity.
		\url{http://www.astro.ruhr-uni-bochum.de/astro/oca/hpt.html}
	\item Atacama Pathfinder EXperiment: APEX is a 12-metre diameter telescope, operating at millimetre and submillimetre wavelengths — between infrared light and 
		radio waves. Submillimetre astronomy opens a window into the cold, dusty and distant Universe, but the faint signals from space are heavily absorbed 
		by water vapour in the Earth's atmosphere. Chajnantor is an ideal location for such a telescope, as the region is one of the driest on the planet and 
		is more than 750 m higher than the observatories on Mauna Kea, and 2400 m higher than the Very Large Telescope (VLT) on Cerro Paranal. 
		\url{http://www.apex-telescope.org/}
	\item SPARTA (only partially): ESO started the development of a common flexible platform called SPARTA for Standard Platform for Adaptive optics Real Time 
		Applications. It's a real time computer and an essentially read-noise free L3 CCD60 provide an ideal cocoon to study the different behavior of the 
		two types of wave front sensors in terms of linearity, sensitivity to calibration errors, noise propagation, specific issues to pyramid or 
		Shack-Hartmann wave front sensors, etc. 
		\url{http://www.eso.org/sci/facilities/develop/ao/tecno/sparta.html}
	\item ARIES21: The ARIES21 40 m radiotelescope is a 40 m diameter fully steerable radiotelescope which operates between 2 and 110 GHz. Located 
		in Yebes, 50 km to the east of Madrid, Spain, it is in operation since 2007. It is part of the EVN (European VLBI Network) and IVS 
		(International VLBI Service). It is operated by the Observatorio Astronómico Nacional, Instituto Geográfico Nacional.
		\url{http://www.oan.es}
\end{itemize}
